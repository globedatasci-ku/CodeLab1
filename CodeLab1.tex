% Options for packages loaded elsewhere
\PassOptionsToPackage{unicode}{hyperref}
\PassOptionsToPackage{hyphens}{url}
%
\documentclass[
]{article}
\usepackage{lmodern}
\usepackage{amssymb,amsmath}
\usepackage{ifxetex,ifluatex}
\ifnum 0\ifxetex 1\fi\ifluatex 1\fi=0 % if pdftex
  \usepackage[T1]{fontenc}
  \usepackage[utf8]{inputenc}
  \usepackage{textcomp} % provide euro and other symbols
\else % if luatex or xetex
  \usepackage{unicode-math}
  \defaultfontfeatures{Scale=MatchLowercase}
  \defaultfontfeatures[\rmfamily]{Ligatures=TeX,Scale=1}
\fi
% Use upquote if available, for straight quotes in verbatim environments
\IfFileExists{upquote.sty}{\usepackage{upquote}}{}
\IfFileExists{microtype.sty}{% use microtype if available
  \usepackage[]{microtype}
  \UseMicrotypeSet[protrusion]{basicmath} % disable protrusion for tt fonts
}{}
\makeatletter
\@ifundefined{KOMAClassName}{% if non-KOMA class
  \IfFileExists{parskip.sty}{%
    \usepackage{parskip}
  }{% else
    \setlength{\parindent}{0pt}
    \setlength{\parskip}{6pt plus 2pt minus 1pt}}
}{% if KOMA class
  \KOMAoptions{parskip=half}}
\makeatother
\usepackage{xcolor}
\IfFileExists{xurl.sty}{\usepackage{xurl}}{} % add URL line breaks if available
\IfFileExists{bookmark.sty}{\usepackage{bookmark}}{\usepackage{hyperref}}
\hypersetup{
  pdftitle={Code Lab 1 Intro},
  pdfauthor={Hannah L. Owens},
  hidelinks,
  pdfcreator={LaTeX via pandoc}}
\urlstyle{same} % disable monospaced font for URLs
\usepackage[margin=1in]{geometry}
\usepackage{color}
\usepackage{fancyvrb}
\newcommand{\VerbBar}{|}
\newcommand{\VERB}{\Verb[commandchars=\\\{\}]}
\DefineVerbatimEnvironment{Highlighting}{Verbatim}{commandchars=\\\{\}}
% Add ',fontsize=\small' for more characters per line
\usepackage{framed}
\definecolor{shadecolor}{RGB}{248,248,248}
\newenvironment{Shaded}{\begin{snugshade}}{\end{snugshade}}
\newcommand{\AlertTok}[1]{\textcolor[rgb]{0.94,0.16,0.16}{#1}}
\newcommand{\AnnotationTok}[1]{\textcolor[rgb]{0.56,0.35,0.01}{\textbf{\textit{#1}}}}
\newcommand{\AttributeTok}[1]{\textcolor[rgb]{0.77,0.63,0.00}{#1}}
\newcommand{\BaseNTok}[1]{\textcolor[rgb]{0.00,0.00,0.81}{#1}}
\newcommand{\BuiltInTok}[1]{#1}
\newcommand{\CharTok}[1]{\textcolor[rgb]{0.31,0.60,0.02}{#1}}
\newcommand{\CommentTok}[1]{\textcolor[rgb]{0.56,0.35,0.01}{\textit{#1}}}
\newcommand{\CommentVarTok}[1]{\textcolor[rgb]{0.56,0.35,0.01}{\textbf{\textit{#1}}}}
\newcommand{\ConstantTok}[1]{\textcolor[rgb]{0.00,0.00,0.00}{#1}}
\newcommand{\ControlFlowTok}[1]{\textcolor[rgb]{0.13,0.29,0.53}{\textbf{#1}}}
\newcommand{\DataTypeTok}[1]{\textcolor[rgb]{0.13,0.29,0.53}{#1}}
\newcommand{\DecValTok}[1]{\textcolor[rgb]{0.00,0.00,0.81}{#1}}
\newcommand{\DocumentationTok}[1]{\textcolor[rgb]{0.56,0.35,0.01}{\textbf{\textit{#1}}}}
\newcommand{\ErrorTok}[1]{\textcolor[rgb]{0.64,0.00,0.00}{\textbf{#1}}}
\newcommand{\ExtensionTok}[1]{#1}
\newcommand{\FloatTok}[1]{\textcolor[rgb]{0.00,0.00,0.81}{#1}}
\newcommand{\FunctionTok}[1]{\textcolor[rgb]{0.00,0.00,0.00}{#1}}
\newcommand{\ImportTok}[1]{#1}
\newcommand{\InformationTok}[1]{\textcolor[rgb]{0.56,0.35,0.01}{\textbf{\textit{#1}}}}
\newcommand{\KeywordTok}[1]{\textcolor[rgb]{0.13,0.29,0.53}{\textbf{#1}}}
\newcommand{\NormalTok}[1]{#1}
\newcommand{\OperatorTok}[1]{\textcolor[rgb]{0.81,0.36,0.00}{\textbf{#1}}}
\newcommand{\OtherTok}[1]{\textcolor[rgb]{0.56,0.35,0.01}{#1}}
\newcommand{\PreprocessorTok}[1]{\textcolor[rgb]{0.56,0.35,0.01}{\textit{#1}}}
\newcommand{\RegionMarkerTok}[1]{#1}
\newcommand{\SpecialCharTok}[1]{\textcolor[rgb]{0.00,0.00,0.00}{#1}}
\newcommand{\SpecialStringTok}[1]{\textcolor[rgb]{0.31,0.60,0.02}{#1}}
\newcommand{\StringTok}[1]{\textcolor[rgb]{0.31,0.60,0.02}{#1}}
\newcommand{\VariableTok}[1]{\textcolor[rgb]{0.00,0.00,0.00}{#1}}
\newcommand{\VerbatimStringTok}[1]{\textcolor[rgb]{0.31,0.60,0.02}{#1}}
\newcommand{\WarningTok}[1]{\textcolor[rgb]{0.56,0.35,0.01}{\textbf{\textit{#1}}}}
\usepackage{graphicx,grffile}
\makeatletter
\def\maxwidth{\ifdim\Gin@nat@width>\linewidth\linewidth\else\Gin@nat@width\fi}
\def\maxheight{\ifdim\Gin@nat@height>\textheight\textheight\else\Gin@nat@height\fi}
\makeatother
% Scale images if necessary, so that they will not overflow the page
% margins by default, and it is still possible to overwrite the defaults
% using explicit options in \includegraphics[width, height, ...]{}
\setkeys{Gin}{width=\maxwidth,height=\maxheight,keepaspectratio}
% Set default figure placement to htbp
\makeatletter
\def\fps@figure{htbp}
\makeatother
\setlength{\emergencystretch}{3em} % prevent overfull lines
\providecommand{\tightlist}{%
  \setlength{\itemsep}{0pt}\setlength{\parskip}{0pt}}
\setcounter{secnumdepth}{-\maxdimen} % remove section numbering

\title{Code Lab 1 Intro}
\author{Hannah L. Owens}
\date{2020-10-09}

\begin{document}
\maketitle

\hypertarget{plan-for-today}{%
\section{Plan for today}\label{plan-for-today}}

\begin{itemize}
\item
  Brief introduction to problem solving in R
\item
  Introduction to tutorial interface
\item ~
  \hypertarget{interactive-problem-solving-exercises}{%
  \subsection{Interactive problem-solving
  exercises!}\label{interactive-problem-solving-exercises}}
\end{itemize}

\hypertarget{problem-solving-in-r}{%
\section{Problem solving in R}\label{problem-solving-in-r}}

.right-column{[}
\includegraphics{https://img.etimg.com/thumb/msid-76642954,width-650,imgsize-330838,,resizemode-4,quality-100/although-a-separate-court-case-established-early-holmes-novels-are-in-the-public-domain-the-lawsuit-alleges-the-detective-only-developed-feelings-in-the-last-10-books-.jpg}{]}
--- \# First, some words of encouragement

\hypertarget{i-can-break-code-all-day-regardless-of-the-data-source.---hannah-owens}{%
\subsubsection{``I can break code all day regardless of the data
source.'' - Hannah
Owens}\label{i-can-break-code-all-day-regardless-of-the-data-source.---hannah-owens}}

\hypertarget{programming-allows-you-to-think-about-thinking-and-while-debugging-you-learn-about-learning.---nicholas-negroponte}{%
\subsubsection{``Programming allows you to think about thinking, and
while debugging you learn about learning.'' - Nicholas
Negroponte}\label{programming-allows-you-to-think-about-thinking-and-while-debugging-you-learn-about-learning.---nicholas-negroponte}}

\hypertarget{the-most-effective-debugging-tool-is-still-careful-thought-coupled-with-judiciously-placed-print-statements.---brian-kerninghan}{%
\subsubsection{``The most effective debugging tool is still careful
thought, coupled with judiciously placed print statements.'' - Brian
Kerninghan}\label{the-most-effective-debugging-tool-is-still-careful-thought-coupled-with-judiciously-placed-print-statements.---brian-kerninghan}}

\hypertarget{some-tips-on-troubleshooting}{%
\section{Some tips on
troubleshooting}\label{some-tips-on-troubleshooting}}

\begin{itemize}
\tightlist
\item
  What is the error?

  \begin{itemize}
  \tightlist
  \item
    Pay attention to line marks
  \item
    Read error messages

    \begin{itemize}
    \tightlist
    \item
      When error message makes no sense, Google it!
    \end{itemize}
  \end{itemize}
\item
  Ask for help

  \begin{itemize}
  \tightlist
  \item
    People you know
  \item
    StackOverflow, GitHub, Reddit, Twitter, etc.
  \end{itemize}
\item
  Walk away (but come back)
\end{itemize}

\hypertarget{some-tips-on-troubleshooting-print-statements}{%
\section{Some tips on troubleshooting: Print
statements}\label{some-tips-on-troubleshooting-print-statements}}

.pull-left{[}

\begin{Shaded}
\begin{Highlighting}[]
\NormalTok{testFunction }\OperatorTok{<}\ErrorTok{*}\StringTok{ }\ControlFlowTok{function}\NormalTok{(y)\{}
  \ControlFlowTok{if}\NormalTok{ (y }\OperatorTok\StringTok{ }\DecValTok{2}\OperatorTok{==}\DecValTok{0}\NormalTok{)\{}
    \KeywordTok{return}\NormalTok{(}\StringTok{"notOdd"}\NormalTok{)}
\NormalTok{  \} }\ControlFlowTok{else} \KeywordTok{return}\NormalTok{()}
\NormalTok{\}}

\NormalTok{x }\OperatorTok{<}\ErrorTok{*}\StringTok{ }\KeywordTok{c}\NormalTok{()}
\ControlFlowTok{for}\NormalTok{(i }\ControlFlowTok{in} \DecValTok{1}\OperatorTok{:}\DecValTok{3}\NormalTok{)\{}
  \KeywordTok{print}\NormalTok{(i)}
  \KeywordTok{print}\NormalTok{(}\StringTok{"executing function"}\NormalTok{)}
\NormalTok{  newI }\OperatorTok{<}\ErrorTok{*}\StringTok{ }\KeywordTok{testFunction}\NormalTok{(i)}
  \KeywordTok{print}\NormalTok{(}\KeywordTok{paste}\NormalTok{(}\StringTok{"newI: "}\NormalTok{, newI))}
  \KeywordTok{print}\NormalTok{(}\StringTok{"appending new i"}\NormalTok{)}
\NormalTok{  x }\OperatorTok{<}\ErrorTok{*}\StringTok{ }\KeywordTok{c}\NormalTok{(x,newI)}
  \KeywordTok{print}\NormalTok{(x)}
\NormalTok{\}}
\end{Highlighting}
\end{Shaded}

{]} .pull-right{[}

\begin{verbatim}
## [1] 1
## [1] "executing function"
## [1] "newI:  "
## [1] "appending new i"
## NULL
## [1] 2
## [1] "executing function"
## [1] "newI:  notOdd"
## [1] "appending new i"
## [1] "notOdd"
## [1] 3
## [1] "executing function"
## [1] "newI:  "
## [1] "appending new i"
## [1] "notOdd"
\end{verbatim}

\hypertarget{section}{%
\subsection{{]}}\label{section}}

\hypertarget{introduction-to-tutorial-interface}{%
\section{Introduction to tutorial
interface}\label{introduction-to-tutorial-interface}}

\hypertarget{introduction-to-tutorial-interface-1}{%
\section{Introduction to tutorial
interface}\label{introduction-to-tutorial-interface-1}}

.pull-left{[} * A tutorial is a special kind of interactive R interface,
powered by Shiny. * Shiny is a framework for building web applications
using R code. * The app is published on the shiny.io server * You don't
even need to have a functioning version of R set up on your machine!{]}

.pull-right{[} {]} --- \# A brief tour of the interface Tutorial app
location: \url{https://hannah-owens.shinyapps.io/CodeLab_1/}

\end{document}
